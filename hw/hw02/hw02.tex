% use the answers clause to get answers to print; otherwise leave it out.
\documentclass[12pts,answers]{exam}
%\documentclass[12pts]{exam}
\RequirePackage{amssymb, amsfonts, amsmath, latexsym, verbatim, xspace, setspace}
\usepackage{graphicx}

% By default LaTeX uses large margins.  This doesn't work well on exams; problems
% end up in the "middle" of the page, reducing the amount of space for students
% to work on them.
\usepackage[margin=1in]{geometry}
\usepackage{enumerate}
\usepackage[hidelinks]{hyperref}

% Here's where you edit the Class, Exam, Date, etc.
\newcommand{\class}{NPRE 555}
\newcommand{\term}{Fall 2020}
\newcommand{\assignment}{HW 2}
\newcommand{\duedate}{2020.09.04}
%\newcommand{\timelimit}{50 Minutes}

\newcommand{\nth}{n\ensuremath{^{\text{th}}} }
\newcommand{\ve}[1]{\ensuremath{\mathbf{#1}}}
\newcommand{\Macro}{\ensuremath{\Sigma}}
\newcommand{\vOmega}{\ensuremath{\hat{\Omega}}}

% For an exam, single spacing is most appropriate
\singlespacing
% \onehalfspacing
% \doublespacing

% For an exam, we generally want to turn off paragraph indentation
\parindent 0ex

%\unframedsolutions

\begin{document} 

% These commands set up the running header on the top of the exam pages
\pagestyle{head}
\firstpageheader{}{}{}
\runningheader{\class}{\assignment\ - Page \thepage\ of \numpages}{Due \duedate}
\runningheadrule

\class \hfill \term \\
\assignment \hfill Due \duedate\\
\rule[1ex]{\textwidth}{.1pt}
%\hrulefill

%%%%%%%%%%%%%%%%%%%%%%%%%%%%%%%%%%%%%%%%%%%%%%%%%%%%%%%%%%%%%%%%%%%%%%%%%%%%%%%%%%%%%
%%%%%%%%%%%%%%%%%%%%%%%%%%%%%%%%%%%%%%%%%%%%%%%%%%%%%%%%%%%%%%%%%%%%%%%%%%%%%%%%%%%%%
\begin{itemize}
        \item Show your work. 
        \item This work must be submitted online as a \texttt{.pdf} through Compass2g.
        \item Work completed with LaTeX or Jupyter earns 1 extra point. Submit 
                source file (e.g. \texttt{.tex} or \texttt{.ipynb}) along with 
                the \texttt{.pdf} file.
        \item If this work is completed with the aid of a numerical program 
                (such as Python, Wolfram Alpha, or MATLAB) all scripts and data 
                must be submitted in addition to the \texttt{.pdf}.
        \item If you work with anyone else, document what you worked on together.
\end{itemize}
\rule[1ex]{\textwidth}{.1pt}

% ---------------------------------------------
\begin{questions}
        \question Prove the relations we mentioned in class:

        \begin{parts}
                \part[5] $n(\vec{r},\vec{\Omega}, E, t) = \left(\frac{v}{m}\right) n(\vec{r},\vec{v},t)$
                \begin{solution}
                        proof here.
                \end{solution}
                \part[5] $n(\vec{r},\vec{\Omega}, E, t) = \left(\frac{1}{mv}\right) n(\vec{r},v, \vec{\Omega},t)$
                \begin{solution}
                        proof here.
                \end{solution}
                \part[5] $n(\vec{r}, v, \vec{\Omega}, t) = v^2  n(\vec{r},\vec{v}, t)$
                \begin{solution}
                        proof here.
                \end{solution}

        \end{parts}

        % ---------------------------------------------
        \question[40] A purely absorbing half-plane medium, in which $\sigma = 1$, 
        contains a sourse emitting $1\left[\frac{n}{cm^3s}\right]$. Determine 
                the intensity and angular distribution of the flux and the 
                current at the surface. (Bell and Glasstone 1.3)
        \begin{solution}
                solution here
        \end{solution}

        % ---------------------------------------------
        \question[45] Suppose there is a purely absorbing region of finite 
        thickness. It is desired to represent this region as an absorbing 
        region accross which the neutron angular density is discontinuous. 
        Derive the discontinuity which is required in the angular density. 
        (Bell and Glasstone 1.10)
        \begin{solution}
                solution here
        \end{solution}

\end{questions}



%\bibliographystyle{plain}
%\bibliography{hw01}
\end{document}
