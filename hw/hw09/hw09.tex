% use the answers clause to get answers to print; otherwise leave it out.
\documentclass[11pt,answers]{exam}
%\documentclass[11pt]{exam}
\RequirePackage{amssymb, amsfonts, amsmath, latexsym, verbatim, xspace, setspace}
\usepackage{graphicx}

% By default LaTeX uses large margins.  This doesn't work well on exams; problems
% end up in the "middle" of the page, reducing the amount of space for students
% to work on them.
\usepackage[margin=1in]{geometry}
\usepackage{enumerate}
\usepackage[hidelinks]{hyperref}

% Here's where you edit the Class, Exam, Date, etc.
\newcommand{\class}{NPRE 555}
\newcommand{\term}{Fall 2020}
\newcommand{\assignment}{HW 9}
\newcommand{\duedate}{2020.12.02}
%\newcommand{\timelimit}{50 Minutes}

\newcommand{\nth}{n\ensuremath{^{\text{th}}} }
\newcommand{\ve}[1]{\ensuremath{\mathbf{#1}}}
\newcommand{\Macro}{\ensuremath{\Sigma}}
\newcommand{\vOmega}{\ensuremath{\hat{\Omega}}}

% For an exam, single spacing is most appropriate
\singlespacing
% \onehalfspacing
% \doublespacing

% For an exam, we generally want to turn off paragraph indentation
\parindent 0ex

%\unframedsolutions

\begin{document} 

% These commands set up the running header on the top of the exam pages
\pagestyle{head}
\firstpageheader{}{}{}
\runningheader{\class}{\assignment\ - Page \thepage\ of \numpages}{Due \duedate}
\runningheadrule

\class \hfill \term \\
\assignment \hfill Due \duedate\\
\rule[1ex]{\textwidth}{.1pt}
%\hrulefill

%%%%%%%%%%%%%%%%%%%%%%%%%%%%%%%%%%%%%%%%%%%%%%%%%%%%%%%%%%%%%%%%%%%%%%%%%%%%%%%%%%%%%
%%%%%%%%%%%%%%%%%%%%%%%%%%%%%%%%%%%%%%%%%%%%%%%%%%%%%%%%%%%%%%%%%%%%%%%%%%%%%%%%%%%%%
\begin{itemize}
        \item Show your work. 
        \item This work must be submitted online as a \texttt{.pdf} through Compass2g.
        \item Work completed with LaTeX or Jupyter earns 1 extra point. Submit 
                source file (e.g. \texttt{.tex} or \texttt{.ipynb}) along with 
                the \texttt{.pdf} file.
        \item If this work is completed with the aid of a numerical program 
                (such as Python, Wolfram Alpha, or MATLAB) all scripts and data 
                must be submitted in addition to the \texttt{.pdf}.
        \item If you work with anyone else, document what you worked on together.
\end{itemize}
\rule[1ex]{\textwidth}{.1pt}

% ---------------------------------------------
\begin{questions}

        % ---------------------------------------------
        \question[30] 
        (Stacey 10.1)
        Calculate the average cosine of the scattering angle in the 
        center-of-mass system for neutrons at 
        1MeV, 100keV, and 1keV colliding with uranium, iron, carbon, and hydrogen.
        \begin{solution}
                solution here
        \end{solution}

        
        % ---------------------------------------------
        \question[40] (Stacey 10.2)
        Calculate the values of the average lethargy increase, $\xi$, and the 
        average cosine of the scattering angle in the lab system, $\mu_0$, for 
        the neutron energies and nuclei of Problem 1 (Stacey 10.1), for 
        isotropic scattering and for linearly isotropic scattering.
        
        \begin{solution}
                solution here
        \end{solution}

        % ---------------------------------------------
        \question[30] (Stacey 12.1)
        Use the proton gas model of Eqs. 12.51 to 12.53 and 12.58 to calculate 
        the low-energy neutron flux distribution in water at 300K. Use 
$\delta_s^H = 38$ barns, $\sigma_s^{H_2O}=0.66$ barn, and $\sigma_s^O=4.2$ 
barns. 
        \begin{solution}
                solution here
        \end{solution}

\end{questions}



%\bibliographystyle{plain}
%\bibliography{hw01}
\end{document}
