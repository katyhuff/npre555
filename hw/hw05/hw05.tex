% use the answers clause to get answers to print; otherwise leave it out.
%\documentclass[12pts,answers]{exam}
\documentclass[12pts]{exam}
\RequirePackage{amssymb, amsfonts, amsmath, latexsym, verbatim, xspace, setspace}
\usepackage{graphicx}

% By default LaTeX uses large margins.  This doesn't work well on exams; problems
% end up in the "middle" of the page, reducing the amount of space for students
% to work on them.
\usepackage[margin=1in]{geometry}
\usepackage{enumerate}
\usepackage[hidelinks]{hyperref}

% Here's where you edit the Class, Exam, Date, etc.
\newcommand{\class}{NPRE 555}
\newcommand{\term}{Fall 2020}
\newcommand{\assignment}{HW 5}
\newcommand{\duedate}{2018.10.12}
%\newcommand{\timelimit}{50 Minutes}

\newcommand{\nth}{n\ensuremath{^{\text{th}}} }
\newcommand{\ve}[1]{\ensuremath{\mathbf{#1}}}
\newcommand{\Macro}{\ensuremath{\Sigma}}
\newcommand{\vOmega}{\ensuremath{\hat{\Omega}}}

% For an exam, single spacing is most appropriate
\singlespacing
% \onehalfspacing
% \doublespacing

% For an exam, we generally want to turn off paragraph indentation
\parindent 0ex

%\unframedsolutions

\begin{document} 

% These commands set up the running header on the top of the exam pages
\pagestyle{head}
\firstpageheader{}{}{}
\runningheader{\class}{\assignment\ - Page \thepage\ of \numpages}{Due \duedate}
\runningheadrule

\class \hfill \term \\
\assignment \hfill Due \duedate\\
\rule[1ex]{\textwidth}{.1pt}
%\hrulefill

%%%%%%%%%%%%%%%%%%%%%%%%%%%%%%%%%%%%%%%%%%%%%%%%%%%%%%%%%%%%%%%%%%%%%%%%%%%%%%%%%%%%%
%%%%%%%%%%%%%%%%%%%%%%%%%%%%%%%%%%%%%%%%%%%%%%%%%%%%%%%%%%%%%%%%%%%%%%%%%%%%%%%%%%%%%
\begin{itemize}
        \item Show your work. 
        \item This work must be submitted online as a \texttt{.pdf} through Compass2g.
        \item Work completed with LaTeX or Jupyter earns 1 extra point. Submit 
                source file (e.g. \texttt{.tex} or \texttt{.ipynb}) along with 
                the \texttt{.pdf} file.
        \item If this work is completed with the aid of a numerical program 
                (such as Python, Wolfram Alpha, or MATLAB) all scripts and data 
                must be submitted in addition to the \texttt{.pdf}.
        \item If you work with anyone else, document what you worked on together.
\end{itemize}
\rule[1ex]{\textwidth}{.1pt}

% ---------------------------------------------
\begin{questions}

        \question In class we examined the general case of neutrons streaming 
        into a vacuum region being emitted from a spherical surface. Here we 
        examine a specific case of streaming outward from a spherical surface 
        into a vacuum region.
        \begin{parts}
        \part[20] Find the angular flux at a position $r$ in the vacuum region 
        if the angular flux on the surface is $\psi(r_s, \mu_s) = 
        \frac{1}{\mu_s}$ for $0 \le \mu_s \le 1$.
        \begin{solution}
                solution here
        \end{solution}

        \part[20] What is the value of the total scalar flux, $\phi(r)$, at the 
        position $r$?
        \begin{solution}
                solution here
        \end{solution}

        \part[10] What angular distribution must a boundary source have to 
        give you the same angular flux distribution in the  vacuum?
        \begin{solution}
                solution here
        \end{solution}
\end{parts}

        % ---------------------------------------------
        \question Consider a purely absorbing slab ($\Sigma_a$).
        \begin{parts}
                \part[10] Using the method of characteristics to derive the 
                streaming term, write the 1-D transport equation in terms of $\mu$, $\Sigma_a$, and source 
                $S(x,\mu)$.

        \begin{solution}
                solution here
        \end{solution}

        \part[10] The slab has length L, vacuum boundaries, and a uniform, 
        isotropic source of volumetric strength $S_0 
        \left[\frac{n}{cm^3}\right]$. What is $S(x, \mu)$ in terms of $S_0$? 

        \begin{solution}
                solution here
        \end{solution}

        \part[10] Find the rightward angular flux $\psi_+$, leftward angular 
        flux $\psi_-$, the scalar flux $\phi(x)$

        \part[20] Plot the angular flux from $0\le x \le L$ for values of $\mu$ corresponding to 
        $\theta = \frac{\pi}{4}$, 
        $\theta = \frac{3\pi}{8}$, 
        $\theta = \frac{3\pi}{4}$, 
        $\theta = \frac{5\pi}{8}$. The plot should contain 4 lines on a single 
        graph. If it is helpful, feel free to choose an explicit length for L.


        \begin{solution}
                solution here
        \end{solution} 
        \end{parts}
\end{questions}



%\bibliographystyle{plain}
%\bibliography{hw01}
\end{document}
