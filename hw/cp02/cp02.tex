% use the answers clause to get answers to print; otherwise leave it out.
%\documentclass[11pt,answers]{exam}
\documentclass[11pt]{exam}
\RequirePackage{amssymb, amsfonts, amsmath, latexsym, verbatim, xspace, setspace}
\usepackage{graphicx}

% By default LaTeX uses large margins.  This doesn't work well on exams; problems
% end up in the "middle" of the page, reducing the amount of space for students
% to work on them.
\usepackage[margin=1in]{geometry}
\usepackage{enumerate}
\usepackage[hidelinks]{hyperref}

% Here's where you edit the Class, Exam, Date, etc.
\newcommand{\class}{NPRE 555}
\newcommand{\term}{Fall 2020}
\newcommand{\assignment}{CP 2}
\newcommand{\duedate}{2020.11.11}
%\newcommand{\timelimit}{50 Minutes}

\newcommand{\nth}{n\ensuremath{^{\text{th}}} }
\newcommand{\ve}[1]{\ensuremath{\mathbf{#1}}}
\newcommand{\Macro}{\ensuremath{\Sigma}}
\newcommand{\vOmega}{\ensuremath{\hat{\Omega}}}

% For an exam, single spacing is most appropriate
\singlespacing
% \onehalfspacing
% \doublespacing

% For an exam, we generally want to turn off paragraph indentation
\parindent 0ex

%\unframedsolutions

\begin{document} 

% These commands set up the running header on the top of the exam pages
\pagestyle{head}
\firstpageheader{}{}{}
\runningheader{\class}{\assignment\ - Page \thepage\ of \numpages}{Due \duedate}
\runningheadrule

\class \hfill \term \\
\assignment \hfill Due \duedate\\
\rule[1ex]{\textwidth}{.1pt}
%\hrulefill

%%%%%%%%%%%%%%%%%%%%%%%%%%%%%%%%%%%%%%%%%%%%%%%%%%%%%%%%%%%%%%%%%%%%%%%%%%%%%%%%%%%%%
%%%%%%%%%%%%%%%%%%%%%%%%%%%%%%%%%%%%%%%%%%%%%%%%%%%%%%%%%%%%%%%%%%%%%%%%%%%%%%%%%%%%%
\begin{itemize}
        \item Show your work. 
        \item This work must be submitted online via github classroom at 
                \url{https://classroom.github.com/a/SpQ5-dUK}. 
        \item All code must be version controlled with git.
        \item If this work is completed with the aid of a numerical program 
                (such as Python, Wolfram Alpha, or MATLAB) all scripts and data 
                must be submitted in addition to the \texttt{.pdf}.
        \item This project should be individual work.
\end{itemize}
\rule[1ex]{\textwidth}{.1pt}

% ---------------------------------------------
\begin{questions}
        \question[30] An infinite bare slab of moderator has thickness 2a. It 
        contains uniformly disributed sources emitting Q $\frac{n}{cm^3s}$. 
        Write a computer program to solve for the flux:
        
        \begin{parts}
                \part Using the $P_1$ approximation and Marshak boundary conditions
                \part Using the $P_1$ approximation and Mark boundary conditions
                \part Using the $P_3$ approximation and Marshak boundary conditions
                \part Using the $P_3$ approximation and Mark boundary conditions
        \end{parts}

        Plot the results to compare the approximations and to compare the 
        boundary conditions.

        \question[30] Report on your results with sufficient clarity to reproduce 
        your work. This should include a clear README describing how your 
        instructor can replicate your flux distribution plot and multiplication 
        factor. A report document (in .pdf format) should include a 4 page description of your method and results.
        Scanned handwritten documents will not be accepted. The report must be generated by a typesetting 
        program (Markdown or LaTeX generated documents are preferred but Word, 
        Open Office, Google Docs are allowed).

	\question Employ good software practices.
        \begin{parts}
                \part[10] Use functions, data structures, and classes appropriately. 
                \part[10] Document your code clearly, using informative 
                variable names, documentation strings, and function call 
                definitions. 
                \part[10] Include a license. Consider BSD-3, a permissive open 
                source licence that requests attribution only.
                \part[10] The code should be well organized, readable, and 
                runnable. 
        \end{parts}

\end{questions}



%\bibliographystyle{plain}
%\bibliography{hw01}
\end{document}
