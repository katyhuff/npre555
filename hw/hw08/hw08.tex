% use the answers clause to get answers to print; otherwise leave it out.
\documentclass[11pt,answers]{exam}
%\documentclass[11pt]{exam}
\RequirePackage{amssymb, amsfonts, amsmath, latexsym, verbatim, xspace, setspace}
\usepackage{graphicx}

% By default LaTeX uses large margins.  This doesn't work well on exams; problems
% end up in the "middle" of the page, reducing the amount of space for students
% to work on them.
\usepackage[margin=1in]{geometry}
\usepackage{enumerate}
\usepackage[hidelinks]{hyperref}

% Here's where you edit the Class, Exam, Date, etc.
\newcommand{\class}{NPRE 555}
\newcommand{\term}{Fall 2020}
\newcommand{\assignment}{HW 8}
\newcommand{\duedate}{2020.11.18}
%\newcommand{\timelimit}{50 Minutes}

\newcommand{\nth}{n\ensuremath{^{\text{th}}} }
\newcommand{\ve}[1]{\ensuremath{\mathbf{#1}}}
\newcommand{\Macro}{\ensuremath{\Sigma}}
\newcommand{\vOmega}{\ensuremath{\hat{\Omega}}}

% For an exam, single spacing is most appropriate
\singlespacing
% \onehalfspacing
% \doublespacing

% For an exam, we generally want to turn off paragraph indentation
\parindent 0ex

%\unframedsolutions

\begin{document} 

% These commands set up the running header on the top of the exam pages
\pagestyle{head}
\firstpageheader{}{}{}
\runningheader{\class}{\assignment\ - Page \thepage\ of \numpages}{Due \duedate}
\runningheadrule

\class \hfill \term \\
\assignment \hfill Due \duedate\\
\rule[1ex]{\textwidth}{.1pt}
%\hrulefill

%%%%%%%%%%%%%%%%%%%%%%%%%%%%%%%%%%%%%%%%%%%%%%%%%%%%%%%%%%%%%%%%%%%%%%%%%%%%%%%%%%%%%
%%%%%%%%%%%%%%%%%%%%%%%%%%%%%%%%%%%%%%%%%%%%%%%%%%%%%%%%%%%%%%%%%%%%%%%%%%%%%%%%%%%%%
\begin{itemize}
        \item Show your work. 
        \item This work must be submitted online as a \texttt{.pdf} through Compass2g.
        \item Work completed with LaTeX or Jupyter earns 1 extra point. Submit 
                source file (e.g. \texttt{.tex} or \texttt{.ipynb}) along with 
                the \texttt{.pdf} file.
        \item If this work is completed with the aid of a numerical program 
                (such as Python, Wolfram Alpha, or MATLAB) all scripts and data 
                must be submitted in addition to the \texttt{.pdf}.
        \item If you work with anyone else, document what you worked on together.
\end{itemize}
\rule[1ex]{\textwidth}{.1pt}

% ---------------------------------------------
\begin{questions}

        % ---------------------------------------------
        \question[30] 
        (Stacey 13.1)
        A critical slab reactor has 1m thickness. 
        Use one speed diffusion theory and perturbation theory to 
        determine the reactivity worth of a 0.25\% increase in the fission 
        cross section over the left half of the slab reactor.
        \begin{solution}
                solution here
        \end{solution}

        
        % ---------------------------------------------
        \question[40] 
        (Stacey 13.2)
        Use two-group diffusion theory and perturbation theory to estimate the 
        reactivity worth of a 0.5\% change in the thermal absoption  cross 
        section of a very large core described by:
        
        \begin{table}[h!]
                \centering
                \begin{tabular}{lll}
                        \hline
                        \textbf{Value} & \textbf{g=1} & \textbf{g=2}\\
                        \hline
                        D & $1.2cm$ &  $0.40cm$\\
                        $\Sigma_a$ & $0.012cm^{-1}$ & $0.120cm^{-1}$ \\
                        $\Sigma_s^{1\rightarrow 2}$& $0.018cm^{-1}$ &  n/a \\
                        $\nu\Sigma_f$ & $0.006cm^{-1}$ & $0.150cm^{-1}$\\
                        \hline
                \end{tabular}
                \caption{Two-Group cross sections in the very large core.}
                \label{tab:prob2}
        \end{table}

        \begin{solution}
                solution here
        \end{solution}


        % ---------------------------------------------
        \question[30]
        (Stacey 13.12)
        Use the Rayleigh quotient to estimate the effective 
        multiplication constant ofr a bare cylindrical core with $\frac{H}{D} 
        =1$, $H=2m$, and one speed diffusion theory parameters $D=1.0cm$, 
        $\Sigma_a=0.15cm^{-1}$, and $\nu\Sigma_f=0.16cm^{-1}$. 
        \begin{solution}
                solution here
        \end{solution}


\end{questions}



%\bibliographystyle{plain}
%\bibliography{hw01}
\end{document}
