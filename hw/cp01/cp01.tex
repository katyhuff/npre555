% use the answers clause to get answers to print; otherwise leave it out.
%\documentclass[12pts,answers]{exam}
\documentclass[12pts]{exam}
\RequirePackage{amssymb, amsfonts, amsmath, latexsym, verbatim, xspace, setspace}
\usepackage{graphicx}

% By default LaTeX uses large margins.  This doesn't work well on exams; problems
% end up in the "middle" of the page, reducing the amount of space for students
% to work on them.
\usepackage[margin=1in]{geometry}
\usepackage{enumerate}
\usepackage[hidelinks]{hyperref}

% Here's where you edit the Class, Exam, Date, etc.
\newcommand{\class}{NPRE 555}
\newcommand{\term}{Spring 2018}
\newcommand{\assignment}{CP 01}
\newcommand{\duedate}{2018.03.07}
%\newcommand{\timelimit}{50 Minutes}

\newcommand{\nth}{n\ensuremath{^{\text{th}}} }
\newcommand{\ve}[1]{\ensuremath{\mathbf{#1}}}
\newcommand{\Macro}{\ensuremath{\Sigma}}
\newcommand{\vOmega}{\ensuremath{\hat{\Omega}}}

% For an exam, single spacing is most appropriate
\singlespacing
% \onehalfspacing
% \doublespacing

% For an exam, we generally want to turn off paragraph indentation
\parindent 0ex

%\unframedsolutions

\begin{document} 

% These commands set up the running header on the top of the exam pages
\pagestyle{head}
\firstpageheader{}{}{}
\runningheader{\class}{\assignment\ - Page \thepage\ of \numpages}{Due \duedate}
\runningheadrule

\class \hfill \term \\
\assignment \hfill Due \duedate\\
\rule[1ex]{\textwidth}{.1pt}
%\hrulefill

%%%%%%%%%%%%%%%%%%%%%%%%%%%%%%%%%%%%%%%%%%%%%%%%%%%%%%%%%%%%%%%%%%%%%%%%%%%%%%%%%%%%%
%%%%%%%%%%%%%%%%%%%%%%%%%%%%%%%%%%%%%%%%%%%%%%%%%%%%%%%%%%%%%%%%%%%%%%%%%%%%%%%%%%%%%
\begin{itemize}
        \item Show your work. 
        \item This work must be submitted online via github classroom at 
                https://classroom.github.com/a/OBRRqht4 .
        \item All code must be version controlled with git.
        \item If this work is completed with the aid of a numerical program 
                (such as Python, Wolfram Alpha, or MATLAB) all scripts and data 
                must be submitted in addition to the \texttt{.pdf}.
        \item This project should be individual work.
\end{itemize}
\rule[1ex]{\textwidth}{.1pt}

% ---------------------------------------------
\begin{questions}
        \question[30] Write a Monte Carlo code to calculate the multiplication
constant and flux distribution for one-speed neutrons in a
slab reactor of thickness $a = 1.0 m$ with isotropic scattering
for which ($\Sigma_a = 0.12 cm^{−1}$, $\Sigma_s = 0.05 cm^{−1}$,
$\nu_f= 0.15 cm^{−1}$) over $0 <x< 50 cm$ and ($\Sigma_a = 0.10 cm^{−1}$,
$s = 0.05 cm^{−1}$, $\nu_f = 0.12$) over $50 <x< 100 cm$.

        \question[30] Report on your results with sufficient clarity to reproduce 
        your work. This should include a clear README describing how your 
        instructor can replicate your flux distribution plot and multiplication 
        factor. A report document (in any format, but markdown or latex are 
        preferred) should include a 4 page description of your method and 
        results.
	\question Employ good software practices.
        \begin{parts}
                \part[10] Use functions, data structures, and classes appropriately. 
                \part[10] Document your code clearly, using informative 
                variable names, documentation strings, and function call 
                definitions. 
                \part[10] Include a license. Consider BSD-3, a permissive open 
                source licence that requests attribution only.
                \part[10] The code should be well organized, readable, and 
                runnable. 
        \end{parts}

\end{questions}



%\bibliographystyle{plain}
%\bibliography{hw01}
\end{document}
