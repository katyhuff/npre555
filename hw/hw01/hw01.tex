% use the answers clause to get answers to print; otherwise leave it out.
\documentclass[12pts,answers]{exam}
%\documentclass[12pts]{exam}
\RequirePackage{amssymb, amsfonts, amsmath, latexsym, verbatim, xspace, setspace}
\usepackage{graphicx}

% By default LaTeX uses large margins.  This doesn't work well on exams; problems
% end up in the "middle" of the page, reducing the amount of space for students
% to work on them.
\usepackage[margin=1in]{geometry}
\usepackage{enumerate}
\usepackage[hidelinks]{hyperref}

% Here's where you edit the Class, Exam, Date, etc.
\newcommand{\class}{NPRE 412}
\newcommand{\term}{Fall 2017}
\newcommand{\assignment}{HW 1}
\newcommand{\duedate}{2017.09.08}
%\newcommand{\timelimit}{50 Minutes}

\newcommand{\nth}{n\ensuremath{^{\text{th}}} }
\newcommand{\ve}[1]{\ensuremath{\mathbf{#1}}}
\newcommand{\Macro}{\ensuremath{\Sigma}}
\newcommand{\vOmega}{\ensuremath{\hat{\Omega}}}

% For an exam, single spacing is most appropriate
\singlespacing
% \onehalfspacing
% \doublespacing

% For an exam, we generally want to turn off paragraph indentation
\parindent 0ex

%\unframedsolutions

\begin{document} 

% These commands set up the running header on the top of the exam pages
\pagestyle{head}
\firstpageheader{}{}{}
\runningheader{\class}{\assignment\ - Page \thepage\ of \numpages}{Due \duedate}
\runningheadrule

\class \hfill \term \\
\assignment \hfill Due \duedate\\
\rule[1ex]{\textwidth}{.1pt}
%\hrulefill

%%%%%%%%%%%%%%%%%%%%%%%%%%%%%%%%%%%%%%%%%%%%%%%%%%%%%%%%%%%%%%%%%%%%%%%%%%%%%%%%%%%%%
%%%%%%%%%%%%%%%%%%%%%%%%%%%%%%%%%%%%%%%%%%%%%%%%%%%%%%%%%%%%%%%%%%%%%%%%%%%%%%%%%%%%%
\begin{itemize}
        \item Show your work. 
        \item This work must be submitted online as a \texttt{.pdf} through Compass2g.
        \item Work completed with LaTeX or Jupyter earns 1 extra point. Submit 
                source file (e.g. \texttt{.tex} or \texttt{.ipynb}) along with 
                the \texttt{.pdf} file.
        \item If this work is completed with the aid of a numerical program 
                (such as Python, Wolfram Alpha, or MATLAB) all scripts and data 
                must be submitted in addition to the \texttt{.pdf}.
        \item If you work with anyone else, document what you worked on together.
\end{itemize}
\rule[1ex]{\textwidth}{.1pt}

% ---------------------------------------------
\begin{questions}
        \question 
        \begin{parts}
                \part[1] What are you hoping to learn in this class?
                \begin{solution}
                        Any relevant answer ok.
                \end{solution}
                \part[1] By what day and time every week must online quizzes be 
                completed?
                \begin{solution}
                        Mondays at 10am.
                \end{solution}
                \part[1] When is homework due every week?
                \begin{solution}
                        Fridays at 10am.
                \end{solution}

        \end{parts}

        % ---------------------------------------------
        \addpoints
        % intro
        \question 
        \begin{parts}
                \part[1] In your opinion, what is the most compelling reason to support nuclear 
                power?
                \begin{solution}
                        Any thuoughtful opinion is ok.
                \end{solution}
                \part[1] In your opinion, what is the most compelling reason to oppose nuclear 
                power?
                \begin{solution}
                        Any thoughtful opinion is ok.
                \end{solution}
        \end{parts}

        %------------------------------------
        %------------------------------------
        \addpoints
        \question 
        Domestic electricity consumption data can be found at \href{http://eia.gov}{eia.gov}.
        \begin{parts}
                \part[5] Using the data you can find on this website, predict the average annual growth rate of domestic (United States) electricity demand for the period between 2015 and 2040. Your answer should be reported as a percent (\%).
                \begin{solution}
                        The correct answer is slightly less than $1\%$. Any answer under $3\%$ should get full 
                        points. Answers far from that number should be docked 
                        one point per percentage difference.  
                        It should be presented as a percentage. If it is 
                        presented in other units, give no credit.
                \end{solution}

                \part[10] Cite your sources, explain your logic, and defend your prediction.
                \begin{solution}
                        \emph{The student should cite one or more EIA documents and provide a logical 
                        path to their prediction.} 

                        Though the historic domestic electricity 
                        demand growth has been higher, it has been levelling off at 
                        approximately $1\%$. The EIA Annual Energy Outlook from 2015 provides a 
                        reference case that predicts this will remain steady 
                        \cite{eia_annual_2015}.
                \end{solution}
        \end{parts}

\addpoints
\question 
International electricity consumption data can be found at \href{http://iea.org}{iea.org}.
\begin{parts}
        \part[5] Using the data you can find on this website, predict the average annual 
        growth rate of worldwide electricity demand for the period between 2015 and 
        2040. Your answer should be reported as a percent (\%).
        \begin{solution}
                The correct answer is around $4\%$. Any answer between $3\%$ and $5\%$ 
                should get full points. Answers far from that number should be docked 
                one point per deviation.  It should be presented as a percentage. If it 
                is presented in other units, it earns 0 points.
        \end{solution}
        \part[10] Cite your sources, explain your logic, and defend your prediction.
        \begin{solution}
                \emph{The student should cite one or more IEA documents and 
                provide a logical path to their prediction.} 

                Though OECD countries have low rates of energy demand growth, 
                non-OECD countries will see a sharp rise. 
                The IEA from 2015 provides a reference case that predicts this will 
                remain steady at 4\% \cite{iea_key_2016}.
        \end{solution}
\end{parts}

%------------------------------------
\addpoints
\question 
\begin{parts}
        \part[5] Do some research using the resources available to you. What was 
        the installed, coal-fired electricity capacity in the United States in 
        \textbf{2014} (in GW)?
        \begin{solution}
                The answer should be about 300 GW, the so-called summer capacity. 
                However, the `nameplate' capacity is 325. So, either 300GW or 
                325 should receive full credit.
        \end{solution}
        \part[5] What is the total installed electricity capacity in the United 
        States in \textbf{2014} (in GW)?
        \begin{solution}
                1100 GW
        \end{solution}
        \part[5] How do the capacity factors of coal-fired plants and nuclear 
        powered plants compare?
        \begin{solution}
                Nuclear power has a capacity factor of 90\%.
                Coal has a capacity factor of 64\%.
        \end{solution}
\end{parts}
%------------------------------------
\addpoints
\question 
\begin{parts}
        \part[10] Using the Electricity Data Browser at eia.gov, what share of 
        electricity \emph{net generation}  was from US coal-fired plants in 
        2014? Please 
        report this to a tenth of a percent resolution.
        \begin{solution}
                For both of these problems, I found, in the data browser, net 
                generation from all sources. 
                Dividing for coal, you get about 33.2\%. (More exactly, 33.18\%)
        \end{solution}
        \part[10] Similarly report the nuclear power share of net electricity 
        generation in 2014. 
        \begin{solution}
                19.5\%. (More exactly, 19.50\%)
        \end{solution}
        %\begin{figure}[h]
        %        \includegraphics[width=\textwidth]{./6.png}
        %        \caption{
        %                Note, the data browser page looked like this when I 
        %                completed the problem concerning net generation.}
        %        \label{fig:eia}
        %\end{figure}


        \part[10] If the United States decided tomorrow to replace all coal-fired 
        electricity generation with nuclear power, how many new nuclear 
        reactors would need to be built to do that?  Making any assumptions you 
        need to concerning the average capacity of new nuclear reactors, but be 
        realistic. 
        \begin{solution}
                Some folks will realize that 19\% of our nation's 
                electricity generation is supplied by 100GW of nuclear reactors. So, 
                33\% of our electricity could be supplied by 
                $100\times\frac{33}{19} = 174$ GW. We need approximately 
                174GW. If you assume reactors are about 1 GW each (which is the 
                case, on average, at the moment in the US),  then we'll need 
                approximately 174 reactors.
        \end{solution}
        \part[10] Assume it takes 10 years to build a nuclear reactor in the United 
        States and the reactor vessel must be installed in year 9. Let us 
        imagine that, unfortunately, builds are constrained by the limited 
        global heavy forging capacity. Assume that only 5 new reactor vessels 
        can be delivered to our shores (from Japan, South Korea, and the UK, 
        primarily) per year. In what year will we have completed the 
        replacement of current coal with nuclear?
        \begin{solution}
                $\frac{174}{5} = 34.8$. It will take at least 35 years. $2016 + 
                35 = 2051$. Add a year because the final pressure vessel must 
                be in place a year before the reactor comes online. That will 
                be the year 2052.  
        \end{solution}
        \part[10] Assuming the domestic electricity demand growth rate you 
        estimated before, what quantity of electricity generation will be 
        needed in the United States that year (in BkWh)?
        \begin{solution}
                 I will use 1\% for this calculation. The student should use 
                 their own estimate without having points taken away. My 
                 calculation is thus: 

                 $F = P(1+i)^n = 3715\times(1.01)^{2052-2015} = 5313$ BkWH. 

                 Stating and using that formula correctly earns full points, 
                 even if the answer is not close due to a bad growth rate 
                 assumption. Close estimates with no work earn 2 points.
                 Bad guesses with no work earn 0 points.
        \end{solution}
\end{parts}

\end{questions}



%\bibliographystyle{plain}
%\bibliography{hw01}
\end{document}
