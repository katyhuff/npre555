% use the answers clause to get answers to print; otherwise leave it out.
\documentclass[12pts,answers]{exam}
%\documentclass[12pts]{exam}
\RequirePackage{amssymb, amsfonts, amsmath, latexsym, verbatim, xspace, setspace}
\usepackage{graphicx}

% By default LaTeX uses large margins.  This doesn't work well on exams; problems
% end up in the "middle" of the page, reducing the amount of space for students
% to work on them.
\usepackage[margin=1in]{geometry}
\usepackage{enumerate}
\usepackage[hidelinks]{hyperref}

% Here's where you edit the Class, Exam, Date, etc.
\newcommand{\class}{NPRE 555}
\newcommand{\term}{Spring 2018}
\newcommand{\assignment}{HW 5}
\newcommand{\duedate}{2018.01.31}
%\newcommand{\timelimit}{50 Minutes}

\newcommand{\nth}{n\ensuremath{^{\text{th}}} }
\newcommand{\ve}[1]{\ensuremath{\mathbf{#1}}}
\newcommand{\Macro}{\ensuremath{\Sigma}}
\newcommand{\vOmega}{\ensuremath{\hat{\Omega}}}

% For an exam, single spacing is most appropriate
\singlespacing
% \onehalfspacing
% \doublespacing

% For an exam, we generally want to turn off paragraph indentation
\parindent 0ex

%\unframedsolutions

\begin{document} 

% These commands set up the running header on the top of the exam pages
\pagestyle{head}
\firstpageheader{}{}{}
\runningheader{\class}{\assignment\ - Page \thepage\ of \numpages}{Due \duedate}
\runningheadrule

\class \hfill \term \\
\assignment \hfill Due \duedate\\
\rule[1ex]{\textwidth}{.1pt}
%\hrulefill

%%%%%%%%%%%%%%%%%%%%%%%%%%%%%%%%%%%%%%%%%%%%%%%%%%%%%%%%%%%%%%%%%%%%%%%%%%%%%%%%%%%%%
%%%%%%%%%%%%%%%%%%%%%%%%%%%%%%%%%%%%%%%%%%%%%%%%%%%%%%%%%%%%%%%%%%%%%%%%%%%%%%%%%%%%%
\begin{itemize}
        \item Show your work. 
        \item This work must be submitted online as a \texttt{.pdf} through Compass2g.
        \item Work completed with LaTeX or Jupyter earns 1 extra point. Submit 
                source file (e.g. \texttt{.tex} or \texttt{.ipynb}) along with 
                the \texttt{.pdf} file.
        \item If this work is completed with the aid of a numerical program 
                (such as Python, Wolfram Alpha, or MATLAB) all scripts and data 
                must be submitted in addition to the \texttt{.pdf}.
        \item If you work with anyone else, document what you worked on together.
\end{itemize}
\rule[1ex]{\textwidth}{.1pt}

% ---------------------------------------------
\begin{questions}
        \question[30] Derive the energy loss relationship for elastic scattering 
        between a neutron and a target atom of atom number A in the 
        \emph{laboratory} reference frame. Assume the neutron scatters through 
        an angle $\theta$.
                \begin{solution}
                        solution here
                \end{solution}

        % ---------------------------------------------
        \question At higher energies, E, the elastic angular differential 
        cross section in the center of mass system ($cm$) exhibits anisotropy 
        (so called p-wave) scattering of the form:
        
        \begin{align*}
                \frac{d\sigma(\theta_{cm})}{d\Omega} = 
                \frac{\sigma_s}{4\pi}(1+a\cos(\theta_{cm})) \text{ where } -1 \le a \le 1
        \end{align*}

        \begin{parts}
                \part[10] Find the elastic cross section at this energy. 
        \begin{solution}
                solution here
        \end{solution}

                
                \part[20] What fraction of the elastically scattered
                neutrons appear at angles greater that 90$^\circ$ 
                in the center-of-mass system?
        \begin{solution}
                solution here
        \end{solution}

                
                \part[30] Plot the differential cross section in the center-of-mass and laboratory 
                systems for the three cases $a = 0.8$, $a = 0$, and $a = -0.8$. 
                Note that one must transform the center-of-mass cross section to 
                the laboratory reference frame 
                \begin{align*}
                \sigma(\theta_{cm})d\hat{\Omega_{cm}} = \sigma_L(\theta_L)d\hat{\Omega_L}
                \end{align*}
        \begin{solution}
                solution here
        \end{solution}
\end{parts}

\end{questions}



%\bibliographystyle{plain}
%\bibliography{hw01}
\end{document}
