\documentclass[11pt, a4paper]{article}
\usepackage[inner=1in,outer=1in,top=1in,bottom=1in]{geometry}
\pagestyle{empty}
\usepackage{placeins}
\usepackage{graphicx}
\usepackage{fancyhdr, lastpage, bbding, pmboxdraw}
\usepackage[usenames,dvipsnames]{color}
\definecolor{darkblue}{rgb}{0,0,.6}
\definecolor{darkred}{rgb}{.7,0,0}
\definecolor{darkgreen}{rgb}{0,.6,0}
\definecolor{red}{rgb}{.98,0,0}
\usepackage[colorlinks,pdfusetitle,urlcolor=darkblue,citecolor=darkblue,linkcolor=darkred,bookmarksnumbered,plainpages=false]{hyperref}
%\renewcommand{\thefootnote}{\fnsymbol{footnote}}

\pagestyle{fancyplain}
\fancyhf{}
\lhead{ \fancyplain{}{\CourseTitle} }
%\chead{ \fancyplain{}{} }
\rhead{ \fancyplain{}{\CourseSemester \CourseYear} }
%\rfoot{\fancyplain{}{page \thepage\ of \pageref{LastPage}}}
\fancyfoot[RO, LE] {page \thepage\ of \pageref{LastPage} }
\thispagestyle{plain}
\usepackage{tabularx}
\usepackage{amsmath}

%%%%%%%%%%%%%%%%%%%%%%%%%%%%%%%%%%%%
\usepackage{xspace}
\newcommand{\CourseNumber}{NPRE555}
\newcommand{\CourseTitle}{Reactor Theory I\xspace}%
\newcommand{\CourseSemester}{Fall\xspace}%
\newcommand{\CourseYear}{2025\xspace}%
\newcommand{\CourseDays}{MW\xspace}%
\newcommand{\CourseStart}{1:00pm\xspace}%
\newcommand{\CourseEnd}{2:50pm\xspace}%
\newcommand{\CourseInstructor}{Prof. Kathryn Huff}
\newcommand{\CourseInstructorEmail}{kdhuff@illinois.edu}
\newcommand{\CourseRoom}{\href{https://maps.app.goo.gl/mv1KTevxS3pmbSim6}{106B3 Engineering Hall}\xspace}%
\newcommand{\CourseBuilding}{\xspace}%
\newcommand{\CourseUniversity}{University of Illinois, Urbana-Champaign\xspace}%
\newcommand{\HuffOffice}{111B Talbot Laboratory\xspace}
\newcommand{\TeachingAssistant}{TA Name\xspace}%
\newcommand{\TAOfficeHourDays}{Wednesdays\xspace}%
\newcommand{\TAOfficeHourStart}{1:00pm\xspace}%
\newcommand{\TAOfficeHourEnd}{3:00pm\xspace}%
\newcommand{\TAOfficeHourPlace}{123 Talbot Laboratory\xspace}
%\newcommand{\Course<++>}{<++>}
%\newcommand{\Course<++>}{<++>}
%%%%%%%%%%%%%%%%%%%%%%%%%%%%%%%%%%%%
\title{\CourseNumber: \CourseTitle\\}
\author{\CourseUniversity}
\date{\CourseSemester \CourseYear}
\begin{document}
\maketitle
%\setlength{\unitlength}{1in}
\renewcommand{\arraystretch}{2}
\begin{center}
\begin{table}[h]
\begin{tabularx}{\textwidth}{rXrX}
\hline
\textbf{Instructor:} & \CourseInstructor & \textbf{Time:} & \CourseDays \CourseStart -- \CourseEnd \\
\textbf{Email:} &  \href{mailto:\CourseInstructorEmail}{\CourseInstructorEmail} & \textbf{Place:} & \CourseRoom \CourseBuilding\\
\hline
\end{tabularx}

\end{table}
\end{center}

\paragraph{Course Pages:}
\begin{enumerate}
        \item \url{https://canvas.illinois.edu/?}
        \item \url{https://github.com/katyhuff/\CourseNumber}
        \item \url{https://mediaspace.illinois.edu/playlist/dedicated/1_fetbf4t5\\}
\end{enumerate}

%\paragraph{TA Office Hours:} The teaching assistant for the course,
%\TeachingAssistant, will hold office hours \TAOfficeHourDays from
%\TAOfficeHourStart to \TAOfficeHourEnd in \TAOfficeHourPlace.

\paragraph{Office Hours:} Prof. Huff will hold office hours by appointment 
only and prefers to meet in her office, \HuffOffice. Please make an appointment at 
\url{https://katyhuff.youcanbook.me}. If your colleagues might be helpful, please 
discuss your questions with them directly before scheduling office hours.

\paragraph{Main References:}
A few essential references for this course will be assigned as readings and 
will be the source of homework problems. The required texts for this course are 
\cite{stacey_nuclear_2007} and \cite{bell_nuclear_1970}. I also recommend 
\cite{duderstadt_transport_1979} for conceptual review and 
\cite{lewis_computational_1993} for computational methods details.
Electronic copies of these books can be found online if they are not available
in the UIUC library.

\bibliographystyle{unsrt}
\renewcommand{\refname}{\normalfont\selectfont\normalsize}\vspace{-1cm}
\bibliography{bibliography}

\paragraph{Objectives:}
\begin{itemize}
\item Advanced development of neutron transport theory
\item  neutron slowing-down,
\item resonance absorption,
\item  approximations to the transport equation,
\item  direct numerical methods and other techniques of approximation theory 
        applied to the neutron transport equation,
\item  and advanced topics. 
\end{itemize}

\paragraph{Prerequisites:}
\begin{itemize}
\item NPRE455 (waived for Physics Majors) 
\end{itemize}

\paragraph{Grading Policy:} Grades will be assigned as a weighted sum of the
following work.

\begin{table}[h]
\begin{tabularx}{\textwidth}{Xr}
        \textbf{Work} & \textbf{Weight} \\
\hline
\textbf{Homework}    & (50\%)  \\
\textbf{Project 1}    & (15\%)  \\
\textbf{Project 2}    & (15\%)  \\
\textbf{Final Project}  & (20\%)  \\
\hline
\textbf{Total}       & (100\%)\\
\end{tabularx}
\end{table}

\paragraph{Important Dates:}
\begin{center} \begin{minipage}{3.8in}
\begin{flushleft}
Project \#1      \dotfill 1:00pm, Monday October 6, 2025 \\
Project \#2      \dotfill 1:00pm, Monday November 10, 2025\\
%Project Deadline \dotfill ~Month Day \\
Final Project    \dotfill 5:00pm, Wednesday Dec. 17, 2025\\
Presentation \dotfill 7:00pm--10:00pm, Wednesday Dec. 17, 2025\\
\end{flushleft}
\end{minipage}
\end{center}

\paragraph{Class Policies:}

\begin{itemize}
\item[] \textbf{Integrity:} This is an institution of higher
learning. You will be swiftly ejected from the course if you are caught
undermining its integrity. Note the
\href{http://www.provost.illinois.edu/academicintegrity/students.html}{Student's
Quick Reference Guide to Academic Integrity} and the
\href{http://studentcode.illinois.edu/article1_part4_1-401.html}{Academic
Integrity Policy and Procedure}.
\item[] \textbf{Attendance:} Regular attendance is mandatory. Request approval
        for absence for extenuating circumstances prior to absence.
                If you are or become ill, please 
                communicate with me and the Dean of Students regarding your 
                expected need for accommodation in this class. I am willing, 
                able, and ready to make neccessary accommodations.
\item[] \textbf{Electronics:} 
        Active participation is essential and expected.
        Accordingly, students must turn off all electronic devices (laptop,
        tablets, cellphones, etc.) during class. Exceptions may be granted for
        tablets or laptops if engaging in computational exercises or taking notes.
\item[] \textbf{Collaboration:} Collaboratively reviewing course materials and 
        studying for exams with fellow students can be enriching.  This is 
                recommended.  However, unless otherwise instructed, homework 
                assignments are to be completed independently and materials 
                submitted as homework should be the result of one's own 
                independent work.
\item[] \textbf{Late Work:} Late work has a halflife of 1 hour. That is, adjusted for lateness, your grade $G(t)$ is a decaying percentage of the raw grade $G_0$. An assignment turned in $t$ hours late will receive a grade according to the following relation:
\begin{align*}
        G(t) &= G_0e^{-\lambda t}
        \intertext{where}
        G(t) &= \mbox{grade adjusted for lateness}\\
        G_0 &= \mbox{raw grade}\\
        \lambda &= \frac{ln(2)}{t_{\frac{1}{2}}} = \mbox{decay constant} \\
        t &= \mbox{time elapsed since due [hours]}\\
        t_{1/2} &= 1 = \mbox{half-life [hours]} \\
\end{align*}
\item[] \textbf{Make-up Work:} There will be no negotiation about late work 
        except in the case of absence documented by an absence letter from the 
                Dean of Students.  The university policy for requesting such a 
                letter is in 
                \href{http://studentcode.illinois.edu/article1_part5_1-501.html}{the 
                Student Code}. Please note that such a letter is appropriate 
                for many types of conflicts, but that religious conflicts 
                require special early handling. In accordance with university 
                policy, students seeking an excused absence for religious 
                reasons should complete the Request for Accommodation for 
                Religious Observances Form, which can be found on the Office of 
                the Dean of Students website. The student should submit this 
                form to the instructor and the Office of the Dean of Students 
                by the end of the second week of the course to which it 
                applies.
\item[]\textbf{Emergency response recommendations} can be found at
        \href{http://police.illinois.edu/em/planning/emergency-response-guide/}{this website}.
        I encourage you to review this website and the campus building
        \href{http://police.illinois.edu/emergency-preparedness/building-emergency-action-plans/}{floor
        plans website} within the first 10 days of class.  If you want to
        better prepare yourself for any of these situations, visit
        \href{http://police.illinois.edu/safe}{police.illinois.edu/safe}. Remember you can sign up for emergency
        text messages at \href{http://emergency.illinois.edu}{emergency.illinois.edu}.
\end{itemize}



\paragraph{Grade Disputes:} It is important that you understand and agree
        with the grade you receive on assignments and exams. If you would like
        to dispute your score, you must send an explanation by email to Prof.
        Huff within one week of recieving the grade.
        \textbf{Do not expect me to regrade anything while in conversation with
        you} as that would not be fair to the other students in the class, whose
        homeworks were graded without them present.  If you request a regrade,
        be aware that the entire assignment will be regraded and is subject to
        double-jeopardy: it is possible that your score will go down.
        Regrade requests should be based on an error on my part (e.g., adding
        up the points incorrectly) or what you suspect is a misunderstanding of
        your work (e.g., arriving at the correct answer using an unexpected
        technique). Regrade requests that argue with the rubric (e.g., ``this is
        wrong, but you took too many points off'') will be returned without
        consideration.
        \textbf{Your work should stand alone.} If an assignment is disorganized or
        ambiguous, and requires an extensive explanation to the grader, you
        will likely still lose points. The homeworks not only evaluate your
        understanding of the material - they also evaluate your ability to
        communicate that understanding clearly and concisely.
\paragraph{Anti-Racism and Inclusivity Statement} The Grainger College of
        Engineering is committed to the creation of an anti-racist, inclusive
        community that welcomes diversity along a number of dimensions,
        including, but not limited to, race, ethnicity and national origins,
        gender and gender identity, sexuality, disability status, class, age,
        or religious beliefs. The College recognizes that we are learning
        together in the midst of the Black Lives Matter movement, that Black,
        Hispanic, and Indigenous voices and contributions have largely either
        been excluded from, or not recognized in, science and engineering, and
        that both overt racism and micro-aggressions threaten the well-being of
        our students and our university community.
        The effectiveness of this course is dependent upon each of us to create
        a safe and encouraging learning environment that allows for the open
        exchange of ideas while also ensuring equitable opportunities and
        respect for all of us. Everyone is expected to help establish and
        maintain an environment where students, staff, and faculty can
        contribute without fear of personal ridicule, or intolerant or
        offensive language. If you witness or experience racism,
        discrimination, micro-aggressions, or other offensive behavior, you are
        encouraged to bring this to the attention of the course director if you
        feel comfortable. You can also report these behaviors to the
        \href{https://diversity.illinois.edu/diversity-campus-culture/belonging-resources/}{Campus
        Belonging portal.}
         Based on your report, Campus Belonging will follow up and reach
        out to students to make sure they have the support they need to be
        healthy and safe. If the reported behavior also violates university
        policy, staff in the Office for Student Conflict Resolution may respond
        as well and will take appropriate action.

\paragraph{Disability-Related Accommodations}
To obtain disability-related academic adjustments and/or auxiliary aids,
students with disabilities must contact the course instructor and the
Disability Resources and Educational Services (DRES) as soon as possible. To
contact DRES, you may visit 1207 S. Oak St., Champaign, call 333-4603, e-mail
disability@illinois.edu or go to
\href{https://www.disability.illinois.edu}{disability.illinois.edu}.  If you
are concerned you have a disability-related condition that is impacting your
academic progress, there are academic screening appointments available that can
help diagnosis a previously undiagnosed disability. You may access these by
visiting the DRES website and selecting ``Request an Academic Screening'' at the
bottom of the page.

\paragraph{Family Educational Rights and Privacy Act (FERPA)}
Any student who has suppressed their directory information pursuant to Family
Educational Rights and Privacy Act (FERPA) should self-identify to the
instructor to ensure protection of the privacy of their attendance in this
course. See \href{https://registrar.illinois.edu/academic-records/ferpa/}{this
link} for more information on FERPA.

\paragraph{Religious Observances}
Illinois law requires the University to reasonably accommodate its students'
religious beliefs, observances, and practices in regard to admissions, class
attendance, and the scheduling of examinations and work requirements. You
should examine this syllabus at the beginning of the semester for potential
conflicts between course deadlines and any of your religious observances. If a
conflict exists, you should notify your instructor of the conflict and follow
the procedure at
\href{https://odos.illinois.edu/community-of-care/resources/students/religious-observances/}{this
link} to request appropriate accommodations. This should be done in the first two weeks of classes.

\paragraph{Sexual Misconduct Reporting Obligation:} The University of Illinois is committed to combating sexual misconduct. Faculty and staff members are required to report any instances of sexual misconduct to the University’s Title IX Office. In turn, an individual with the Title IX Office will provide information about rights and options, including accommodations, support services, the campus disciplinary process, and law enforcement options.

A list of the designated University employees who, as counselors, confidential
advisors, and medical professionals, do not have this reporting responsibility
and can maintain confidentiality, can be found
\href{https://wecare.illinois.edu/resources/students/#confidential}{here}.
Other information about resources and reporting is available here: wecare.illinois.edu.

\paragraph{Other Resources:}
University students typically experience a wide range of stressors during their
time on campus. Accordingly, campus resources exist to help students manage
stress levels, mental health, physical health, and emergencies while navigating
this environment. I hope you will take advantage of these campus resources as
soon as they can be of help.

\begin{itemize}
\item \href{https://campusrec.illinois.edu/}{The Campus Recreational Centers}
\item \href{http://counselingcenter.illinois.edu/}{The Counselling Center}
\item \href{http://www.mckinley.illinois.edu/clinics/mental\_health.htm}{The McKinley Mental Health Clinic}
\item \href{http://odos.illinois.edu/emergency/}{The Emergency Dean}
\end{itemize}


\pagebreak
\FloatBarrier
\renewcommand{\arraystretch}{1}
\begin{table}[h]
\begin{center}
\begin{tabular}{lllcllll}
\multicolumn{8}{c}{\textbf{Course Schedule:}\textit{ Note that this schedule is subject to change}}\\
&&&&&&&\\
\textbf{Date} & \textbf{Week} & \textbf{Day} & \textbf{Unit} & \textbf{Chap.} & \textbf{Chap.} & \textbf{HW} & \textbf{HW}\\
              &  &  & & \textbf{Stacey}& \textbf{B\&G} & \textbf{Given} & \textbf{Due}\\
\hline
\hline
08-25 & 1 & M  & Transport Eqn.          & 9    & 1    & HW1 &     \\
08-27 & 1 & W  & Transport Eqn.          & 9    & 1    & CP1 &     \\
09-01 & 2 & M  & Boundary Conds.         & 9    & 1    & HW2 & HW1 \\
09-03 & 2 & W  & Boundary Conds.         & 9    & 1    &     &     \\
09-08 & 3 & M  & Angle \& Energy         & 9    & 1    & HW3 & HW2 \\
09-10 & 3 & W  & Angle \& Energy         & 9    & 1    &     &     \\
09-15 & 4 & M  & Monte Carlo             & 9.12 & 1\&2 & HW4 & HW3 \\
09-17 & 4 & W  & Monte Carlo             & 9.12 & 1\&2 &     &     \\
09-22 & 5 & M  & Monte Carlo             & 9.12 & 1\&2 & HW5 & HW4 \\
09-24 & 5 & W  & Monte Carlo             & 9.12 & 1\&2 &     &     \\
09-29 & 6 & M  & $P_N$                   & 9    & 3    & HW6 & HW5 \\
10-01 & 6 & W  & $P_N$                   & 9    & 3    &     &     \\
10-06 & 7 & M  & Multi-group $P_N$       & 9    & 4    & CP2 & CP1 \\
10-08 & 7 & W  & Multi-group $P_N$       & 9    & 4    &     &     \\
10-13 & 8 & M  & $P_N$ Eigenvalue Calcs. & 9    & 4    & HW7 & HW6 \\
10-15 & 8 & W  & $P_N$ Eigenvalue Calcs. & 9    & 4    &     &     \\
10-20 & 9 & M  & $S_N$                   & 9    & 5    & HW8 & HW7 \\
10-22 & 9 & W  & $S_N$                   & 9    & 5    &     &     \\
10-27 & 10 & M & $S_N$                   & 9    & 5    & HW9 & HW8 \\
10-29 & 10 & W & $S_N$                   & 9    & 5    &     &     \\
11-03 & 11 & M & Adjoint Eqn.            & 13   & 6    & HW10& HW9 \\
11-05 & 11 & W & Adjoint Eqn.            & 13   & 6    &     &     \\
11-10 & 12 & M & Perturbation Theory     & 13   & 6    & CP3 & CP2 \\
11-12 & 12 & W & Variational Methods     &      &      &     &     \\
11-17 & 13 & M & Neutron Slowing Down    & 10   & 7    & HW11& HW10\\
11-19 & 13 & W & Neutron Slowing Down    & 10   & 7    &     &     \\
11-24 & 14 & M & \textbf{Fall Break}     &      &      &     &     \\
11-26 & 14 & W & \textbf{Fall Break}     &      &      &     &     \\
12-01 & 15 & M & Neutron Thermalization  & 11   & 7    & HW12& HW11\\
12-03 & 15 & W & Neutron Thermalization  & 11   & 7    &     &     \\
12-08 & 16 & M & Resonance Absorption    & 12   & 11   &     & HW12\\
12-10 & 16 & W & Resonance Absorption    & 12   & 11   &     &     \\
12-17 & 17 & W & \textbf{Final Presentation} &  &      & CP3 &     \\
\end{tabular}
\end{center}
\end{table}
%%%%%% THE END 
\end{document} 
