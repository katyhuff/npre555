\documentclass[11pt, a4paper]{article}
\usepackage[inner=1in,outer=1in,top=1in,bottom=1in]{geometry}
\pagestyle{empty}
\usepackage{placeins}
\usepackage{graphicx}
\usepackage{fancyhdr, lastpage, bbding, pmboxdraw}
\usepackage[usenames,dvipsnames]{color}
\definecolor{darkblue}{rgb}{0,0,.6}
\definecolor{darkred}{rgb}{.7,0,0}
\definecolor{darkgreen}{rgb}{0,.6,0}
\definecolor{red}{rgb}{.98,0,0}
\usepackage[colorlinks,pdfusetitle,urlcolor=darkblue,citecolor=darkblue,linkcolor=darkred,bookmarksnumbered,plainpages=false]{hyperref}
%\renewcommand{\thefootnote}{\fnsymbol{footnote}}

\pagestyle{fancyplain}
\fancyhf{}
\lhead{ \fancyplain{}{\CourseTitle} }
%\chead{ \fancyplain{}{} }
\rhead{ \fancyplain{}{\CourseSemester \CourseYear} }
%\rfoot{\fancyplain{}{page \thepage\ of \pageref{LastPage}}}
\fancyfoot[RO, LE] {page \thepage\ of \pageref{LastPage} }
\thispagestyle{plain}
\usepackage{tabularx}
\usepackage{amsmath}

%%%%%%%%%%%%%%%%%%%%%%%%%%%%%%%%%%%%
\usepackage{xspace}
\newcommand{\CourseNumber}{NPRE555}
\newcommand{\CourseTitle}{Reactor Theory I\xspace}%
\newcommand{\CourseSemester}{Fall\xspace}%
\newcommand{\CourseYear}{2020\xspace}%
\newcommand{\CourseDays}{MW\xspace}%
\newcommand{\CourseStart}{9:00am\xspace}%
\newcommand{\CourseEnd}{10:50am\xspace}%
\newcommand{\CourseInstructor}{Prof. Kathryn Huff}
\newcommand{\CourseInstructorEmail}{kdhuff@illinois.edu}
\newcommand{\CourseRoom}{\href{https://illinois.zoom.us/j/96634585623?pwd=U1FTL0VJaHh1UjVpRWlGQk1VT3d5dz09}{Zoom: 966 3458 5623}\xspace}%
\newcommand{\CourseBuilding}{\xspace}%
\newcommand{\CourseUniversity}{University of Illinois, Urbana-Champaign\xspace}%
\newcommand{\TeachingAssistant}{TA Name\xspace}%
\newcommand{\TAOfficeHourDays}{Wednesdays\xspace}%
\newcommand{\TAOfficeHourStart}{1:00pm\xspace}%
\newcommand{\TAOfficeHourEnd}{3:00pm\xspace}%
\newcommand{\TAOfficeHourPlace}{123 Talbot Laboratory\xspace}
%\newcommand{\Course<++>}{<++>}
%\newcommand{\Course<++>}{<++>}
%%%%%%%%%%%%%%%%%%%%%%%%%%%%%%%%%%%%
\title{\CourseNumber: \CourseTitle\\}
\author{\CourseUniversity}
\date{\CourseSemester \CourseYear}
\begin{document}
\maketitle
%\setlength{\unitlength}{1in}
\renewcommand{\arraystretch}{2}
\begin{center}
\begin{table}[h]
\begin{tabularx}{\textwidth}{rXrX}
\hline
\textbf{Instructor:} & \CourseInstructor & \textbf{Time:} & \CourseDays \CourseStart -- \CourseEnd \\
\textbf{Email:} &  \href{mailto:\CourseInstructorEmail}{\CourseInstructorEmail} & \textbf{Place:} & \CourseRoom \CourseBuilding\\
\hline
\end{tabularx}

\end{table}
\end{center}

\paragraph{Course Pages:}
\begin{enumerate}
        \item \url{https://compass2g.illinois.edu}
        \item \url{https://github.com/katyhuff/\CourseNumber}
        \item \url{https://mediaspace.illinois.edu/channel//177052391}
\end{enumerate}

%\paragraph{TA Office Hours:} The teaching assistant for the course,
%\TeachingAssistant, will hold office hours \TAOfficeHourDays from
%\TAOfficeHourStart to \TAOfficeHourEnd in \TAOfficeHourPlace.

\paragraph{Office Hours:} Prof. Huff will hold office hours by appointment only
and prefers to meet online this semester. Please make an appointment at 
\url{https://katyhuff.youcanbook.me}. If your colleagues might be helpful, please 
discuss your questions with them directly before scheduling office hours.

\paragraph{Main References:}
A few essential references for this course will be assigned as readings and 
will be the source of homework problems. The required texts for this course are 
\cite{stacey_nuclear_2007} and \cite{bell_nuclear_1970}. I also recommend 
\cite{duderstadt_transport_1979} for conceptual review and 
\cite{lewis_computational_1993} for computational methods details.
Electronic copies of these books can be found online if they are not available
in the UIUC library.

\bibliographystyle{unsrt}
\renewcommand{\refname}{\normalfont\selectfont\normalsize}\vspace{-1cm}
\bibliography{bibliography}

\paragraph{Objectives:}
\begin{itemize}
\item Advanced development of neutron transport theory
\item  neutron slowing-down,
\item resonance absorption,
\item  approximations to the transport equation,
\item  direct numerical methods and other techniques of approximation theory 
        applied to the neutron transport equation,
\item  and advanced topics. 
\end{itemize}

\paragraph{Prerequisites:}
\begin{itemize}
\item NPRE455 (waived for Physics Majors) 
\end{itemize}

\paragraph{Grading Policy:} Grades will be assigned as a weighted sum of the
following work.

\begin{table}[h]
\begin{tabularx}{\textwidth}{Xr}
        \textbf{Work} & \textbf{Weight} \\
\hline
\textbf{Homework}    & (40\%)  \\
\textbf{Project 1}    & (20\%)  \\
\textbf{Project 2}    & (20\%)  \\
\textbf{Final Project}  & (20\%)  \\
\hline
\textbf{Total}       & (100\%)\\
\end{tabularx}
\end{table}

\paragraph{Important Dates:}
\begin{center} \begin{minipage}{3.8in}
\begin{flushleft}
%Midterm \#1      \dotfill 10:00-10:50pm, 3, 2017 \\
%Midterm \#2      \dotfill 10:00-10:50pm, 7, 2017\\
%Project Deadline \dotfill ~Month Day \\
Final Project    \dotfill 8:00am-11:00am., Monday Dec. 14, 2020\\
\end{flushleft}
\end{minipage}
\end{center}

\paragraph{Class Policies:}

\begin{itemize}
\item[] \textbf{Integrity:} This is an institution of higher
learning. You will be swiftly ejected from the course if you are caught
undermining its integrity. Note the
\href{http://www.provost.illinois.edu/academicintegrity/students.html}{Student's
Quick Reference Guide to Academic Integrity} and the
\href{http://studentcode.illinois.edu/article1_part4_1-401.html}{Academic
Integrity Policy and Procedure}.
\item[] \textbf{Attendance:} Attendence is strongly encouraged. While it is 
        typically mandatory, the COVID-19 pandemic requires flexibility. 
                Students unable to attend synchronous class sessions must 
                request recordings of those sessions from Prof. Huff and are 
                expected to watch them. If you are or become ill, please 
                communicate with me and the Dean of Students regarding your 
                expected need for accommodation in this class. I am willing, 
                able, and ready to make neccessary accommodations.
\item[] \textbf{Electronics:} Active participation is essential and expected.
\item[] \textbf{Collaboration:} Collaboratively reviewing course materials and studying for exams with fellow students can be enriching.  This is recommended.  However, unless otherwise instructed, homework assignments are to be completed independently and materials submitted as homework should be the result of one's own independent work.
\item[] \textbf{Late Work:} Extensions due to the pandemic will be considered. 
        However, in general, late work has a halflife of 1 hour. That is, adjusted for lateness, your grade $G(t)$ is a decaying percentage of the raw grade $G_0$. An assignment turned in $t$ hours late will receive a grade according to the following relation:
\begin{align*}
        G(t) &= G_0e^{-\lambda t}
        \intertext{where}
        G(t) &= \mbox{grade adjusted for lateness}\\
        G_0 &= \mbox{raw grade}\\
        \lambda &= \frac{ln(2)}{t_{\frac{1}{2}}} = \mbox{decay constant} \\
        t &= \mbox{time elapsed since due [hours]}\\
        t_{1/2} &= 1 = \mbox{half-life [hours]} \\
\end{align*}
\item[] \textbf{Make-up Work:} There will be no negotiation about late work 
        except in the case of absence documented by an absence letter from the 
                Dean of Students.  The university policy for requesting such a 
                letter is in 
                \href{http://studentcode.illinois.edu/article1_part5_1-501.html}{the 
                Student Code}. Please note that such a letter is appropriate 
                for many types of conflicts, but that religious conflicts 
                require special early handling. In accordance with university 
                policy, students seeking an excused absence for religious 
                reasons should complete the Request for Accommodation for 
                Religious Observances Form, which can be found on the Office of 
                the Dean of Students website. The student should submit this 
                form to the instructor and the Office of the Dean of Students 
                by the end of the second week of the course to which it 
                applies.

\end{itemize}

\paragraph{Accessibility:} I hope that this course will be inclusive and
accommodating for all learners. As such, I am committed upholding the vision
and values of \href{http://www.inclusiveillinois.illinois.edu/index.html}{Inclusive Illinois}
in my
classroom.  With regard to accommodating all learners, please note that many
resources are provided through
\href{http://disability.illinois.edu/academic-support/accommodations}{the
Division of Disability Resources and Educational Services}.  To request
particular accommodations, please contact me as soon as possible so that we can
work out any necessary arrangements.

\paragraph{Other Resources:}
University students typically experience a wide range of stressors during their
time on campus. Accordingly, campus resources exist to help students manage
stress levels, mental health, physical health, and emergencies while navigating
this environment. I hope you will take advantage of these campus resources as
soon as they can be of help.

\begin{itemize}
\item \href{https://campusrec.illinois.edu/}{The Campus Recreational Centers}
\item \href{http://counselingcenter.illinois.edu/}{The Counselling Center}
\item \href{http://www.mckinley.illinois.edu/clinics/mental\_health.htm}{The McKinley Mental Health Clinic}
\item \href{http://odos.illinois.edu/emergency/}{The Emergency Dean}
\end{itemize}

\pagebreak
\FloatBarrier
\renewcommand{\arraystretch}{1}
\begin{table}[h]
\begin{center}
\begin{tabular}{lllcllll}
\multicolumn{8}{c}{\textbf{Course Schedule:}\textit{ Note that this schedule is subject to change}}\\
&&&&&&&\\
\textbf{Date} & \textbf{Week} & \textbf{Day} & \textbf{Unit} & \textbf{Chap.} & \textbf{Chap.} & \textbf{HW} & \textbf{HW}\\
              &  &  & & \textbf{Stacey}& \textbf{B\&G} & \textbf{Given} & \textbf{Due}\\
\hline
\hline
08-24 & 1 & M & Transport Eqn. & 9 & 1 & HW1 & \\
08-26 & 1 & W & Transport Eqn. & 9 & 1 &  & \\
08-31 & 2 & M & Boundary Conds. & 9 & 1 & HW2 & HW1\\
09-02 & 2 & W & Boundary Conds. & 9 & 1 &  & \\
09-07 & 3 & M & no class &  &  &  & \\
09-09 & 3 & W & Angle \& Energy & 9 & 1 &  & \\
09-14 & 4 & M & Angle \& Energy & 9 & 1 & CP1 & \\
09-16 & 4 & W & MC    & 9.12 & 1\&2 &  HW3 & HW2\\
09-21 & 5 & M & MC & 9.12 & 1\&2 &  & \\
09-23 & 5 & W & MC & 9.12 & 1\&2 & HW4 & HW3\\
09-28 & 6 & M & $P_N$ & 9 & 3 &  & \\
09-30 & 6 & W & $P_N$ & 9 & 3 & HW5 & HW4 \\
10-05 & 7 & M & Multi-group $P_N$ & 9 & 4 &  & \\
10-07 & 7 & W & Multi-group $P_N$ & 9 & 4 & CP2 & CP1\\
10-12 & 8 & M & $P_N$ Eigenvalue Calcs.& 9 & 4 &  &   \\
10-14 & 8 & W & $P_N$ Eigenvalue Calcs. & 9 & 4 & HW6 & HW5\\
10-19 & 9 & M & $S_N$ & 9 & 5 &  & \\
10-21 & 9 & W & $S_N$ & 9 & 5 &  & \\
10-26 & 10 & M & $S_N$ & 9 & 5 &     &    \\
10-28 & 10 & W & $S_N$ & 9 & 5 & HW7 & HW6 \\
11-02 & 11 & M & Adjoint Eqn. & 13 & 6 &  & \\
11-04 & 11 & W & Adjoint Eqn. & 13 & 6 &  & \\
11-09 & 12 & M & Perturbation Theory & 13 & 6 & & \\
11-11 & 12 & W & Variational Methods &  &  & CP3 & CP2\\
11-16 & 13 & M & Neutron Slowing Down & 10 &7 &  & \\
11-18 & 13 & W & Neutron Slowing Down & 10 & 7 & HW8 & HW7\\
11-23 & 14 & M & \textbf{Fall Break} &  &  &  & \\
11-25 & 14 & W & \textbf{Fall Break} &  &  &  &  \\
11-30 & 15 & M & Neutron Thermalization &  11 & 7 &  & \\
12-02 & 15 & W & Neutron Thermalization & 11 & 7 & HW9 & HW8\\
12-07 & 16 & M & Resonance Absorption & 12 & 11 &  & \\
12-09 & 16 & W & Resonance Absorption & 12 & 11 &  & HW9 \\
12-14 & 17 & M & \textbf{Final Exam} &  &  & CP3 & \\
\end{tabular}
\end{center}
\end{table}
%%%%%% THE END 
\end{document} 
